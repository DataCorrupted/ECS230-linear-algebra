\documentclass{article}

\title{ECS 230 App Num Linear Algebra \\ Homework 2}
\author{Yuyang(Peter) Rong \\917781535 \\ PtrRong@ucdavis.edu}

\usepackage[utf8]{inputenc}
\usepackage{graphicx}
\usepackage[colorlinks,linkcolor=red]{hyperref}
\usepackage{amsmath, amsthm, amssymb}
\usepackage{subfloat}
\newtheorem{prop}{Proposition}
\usepackage{ulem}
\usepackage{listings}


% For SI numerical display.
\usepackage{siunitx}
\sisetup{detect-all}
% For table \*rules
\usepackage{booktabs}
% To fix table places.
\usepackage{float}
\restylefloat{table}

\newcommand{\norm}[1]{\left\lVert#1\right\rVert}

\begin{document}
\maketitle

\begin{description}
	\item[Problem 1] Error Analysis
	      Consider the problem of solving $Ax = b$ for the unknown $x \in \mathbb{R}^2$ with
	      \begin{equation}
		      A = \begin{pmatrix} a & (a-1) \\ (a-1) & (a-2)\end{pmatrix}
	      \end{equation}
	      where $a \in \mathbb{R}$ and $a > 2$. Assuming that the relative error in $b$ is bounded by $\epsilon > 0$
	      \begin{equation}
		      \frac{\norm{\delta b}_\infty}{\norm{b}_\infty} < \epsilon
	      \end{equation}
	      compute a bound for the relative error:
	      \begin{equation}
		      \frac{\norm{\delta x}_\infty}{\norm{x}_\infty} < \epsilon
	      \end{equation}

	      \begin{proof}
		      Since $a > 2$, It's safe to conclude that:
		      $$
			      A^{-1} = \begin{pmatrix} -(a-2) & (a-1) \\ (a-1) & -a\end{pmatrix}
		      $$
	      \end{proof}

	\item[Problem 2] Using timing functions

	      a) Study and understand the program \lstinline{timing1.c}.
	      Compile the program on CSIF using gcc -o timing1 timing1.c -lm.
	      Run the program 10 times with a command line argument of 5000000. Report the timings observed in the output. Repeat the process with arguments 10000000 and 20000000.
	      Discuss the results observed, in particular regarding the resolution of the timing procedure used and the reproducibility of the results between runs. From the observed results,
	      estimate the time needed to compute a square root.

	      \textbf{a)}
	      Estimated time to compute a square root:

	      \begin{table}[H]
		      \centering
		      \begin{tabular}{SSSS}
			      \toprule
			      \multicolumn{1}{l}{Argument} & {Wall time} & {CPU Time} & {CPU cycles} \\
			      \midrule
			      5000000                      &             &            &              \\
			      10000000                     &             &            &              \\
			      20000000                     &             &            &              \\
			      \bottomrule
		      \end{tabular}
		      \caption{10 time running results for problem a)}
	      \end{table}


	      b) Study and understand the program \lstinline{timing2.c}.
	      Compile the program on CSIF using gcc as was done in a). Run the program 10 times
	      with the same arguments as in case a). Discuss the results obtained. Find the clock rate
	      of the processor by looking at the file /proc/cpuinfo. Using your timing information,
	      compute the number of clock cycles needed to compute a square root and the corresponding time. Compare with the results obtained in a).

	      \textbf{b)}
	      Estimated time to compute a square root:

	      \begin{table}[H]
		      \centering
		      \begin{tabular}{SSSS}
			      \toprule
			      \multicolumn{1}{l}{Argument} & {Wall time} & {CPU Time} & {CPU cycles} \\
			      \midrule
			      5000000                      &             &            &              \\
			      10000000                     &             &            &              \\
			      20000000                     &             &            &              \\
			      \bottomrule
		      \end{tabular}
		      \caption{10 time running results for problem b)}
	      \end{table}

\end{description}

\end{document}
\grid
