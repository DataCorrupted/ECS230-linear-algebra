\documentclass{article}

\title{ECS 230 App Num Linear Algebra \\ Homework 1}
\author{Yuyang(Peter) Rong \\917781535 \\ PtrRong@ucdavis.edu}

\usepackage[utf8]{inputenc}
\usepackage{graphicx}
\usepackage[colorlinks,linkcolor=red]{hyperref}
\usepackage{amsmath, amsthm, amssymb}
\usepackage{subfloat}
\newtheorem{prop}{Proposition}
\usepackage{ulem}
\begin{document}
\maketitle

\begin{description}
	\item[Problem 1] Let $A, B \in \mathbb{R}^{n\times n}$ and assuming both $A^{-1}$ $B^{-1}$ exist. Show that $(AB)^{-1} =
		      B^{-1} A^{-1}$ .
	      \begin{proof}
		      \begin{align*}
			       &                 & I            & =  I                 \\
			       & \Leftrightarrow & I            & =  AB(AB)^{-1}       \\
			       & \Leftrightarrow & A^{-1}       & =  A^{-1}AB(AB)^{-1} \\
			       & \Leftrightarrow & A^{-1}       & =  B(AB)^{-1}        \\
			       & \Leftrightarrow & B^{-1}A^{-1} & =  B^{-1}B(AB)^{-1}  \\
			       & \Leftrightarrow & B^{-1}A^{-1} & =  (AB)^{-1}         \\
		      \end{align*}
	      \end{proof}

	\item[Problem 2] Let $A \in \mathbb{R}^{n\times n}$. Show that if $Ax = 0,  \forall x \in \mathbb{R}^n$ , then $A = 0$.
	      \begin{proof}
		      Proof by contradiction. Suppose $A \ne 0$, then $A$ must have at least one \textbf{non-zero} column.

		      Suppose $A$ has $k$ non-zero column, $a_{\_idx_1}, a_{\_idx_2} \cdots a_{\_idx_k}$, where $idx_i$ denotes the index of that column.

		      $\forall x \in \mathbb{R}^n$, index $i$, $Ax = \sum_{i = 1}{k}x_{idx_i}a_{\_idx_i}=0$

		      Then we must have $\forall i, a_{\_idx_i} = 0$, which contracdicts with our assumption that $A$ has non-zero columns.

		      Therefore, $A = 0$.
	      \end{proof}

	\item[Problem 3] Let $A \in \mathbb{R}^{n\times n}$. Show that if $A$ is strictly lower triangular(i.e. $a_{ij} = 0$ if $i \le j$), then $A^n=0$.
	      \begin{proof}
		      Proof by induction.

		      \begin{itemize}
			      \item When $n = 1$, $A_1 = 0$, $A^1 = 0$
			      \item When $n = k$, suppose $A_k^k = 0$, proof that for $n = k+1$ it is still true.

			            Suppose $A_{k+1} = \begin{bmatrix}
					            A_k               & \textit{v} \\
					            (0, 0, \cdots, 0) & 0
				            \end{bmatrix}	$

			            Since
			            $$\begin{bmatrix}
					            A & B \\
					            0 & 0
				            \end{bmatrix} * \begin{bmatrix}
					            C & D \\
					            0 & 0
				            \end{bmatrix} = \begin{bmatrix}
					            AC & AD \\
					            0  & 0
				            \end{bmatrix}
			            $$
			            and we have:
			            $$A_{k+1} * A_{k+1} = \begin{bmatrix}
					            A_k^2             & A_k\textit{v} \\
					            (0, 0, \cdots, 0) & 0
				            \end{bmatrix}	$$
			            So similarly we can have:
			            $$A_{k+1}^{k+1} = \begin{bmatrix}
					            A_k^k * A_k       & A_k^k\textit{v} \\
					            (0, 0, \cdots, 0) & 0
				            \end{bmatrix} = \begin{bmatrix}
					            0 * A_k           & 0 * \textit{v} \\
					            (0, 0, \cdots, 0) & 0
				            \end{bmatrix} = 0	$$
			      \item Therefore it's safe to conclude that $\forall n$, if $A$ is strictly lower triangular, $A^n = 0$
		      \end{itemize}
	      \end{proof}
	\item[Problem 4] Let $A \in \mathbb{R}^{n\times n}, \textit{u}, \textit{v} \in \mathbb{R}^n$. Compute the inverse of $A = I + \textit{u}\textit{v}^T$.
	      Hint:
	      try a matrix $I - \alpha \textit{u} \textit{v}^T$ with $\alpha \in \mathbb{R}$ to be determined.
	      What constraint must $\textit{u}$ and $\textit{v}$ satisfy for the inverse to exist?
	      \begin{proof}

	      \end{proof}
	\item[Problem 5] \textbf{Using the results obtained in 1) and 4)} and assuming $A \in \mathbb{R}^{n\times n}, \textit{u}, \textit{v} \in \mathbb{R}^n$, and $A^{-1}$ exists. Show that:
	      $$ (A+\textit{u}\textit{v}^T)^{-1} = A^{-1} - \frac{1}{1 + \textit{v}^TA^{-1}\textit{u}}A^{-1}\textit{u}\textit{v}^TA^{-1}$$
	      \begin{proof}

	      \end{proof}
\end{description}

\end{document}
\grid
