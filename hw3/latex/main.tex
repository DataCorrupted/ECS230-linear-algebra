\documentclass[12pt]{article}
\usepackage[margin=1in]{geometry}
\usepackage{mdframed}
\usepackage{booktabs}

% For math
\usepackage{amsmath, amsthm, amssymb}

% For coding
\usepackage{listings}

% For SI numerical display.
\usepackage{siunitx}
\sisetup{detect-all}

% For table \*rules
\usepackage{booktabs}

% To fix table places.
\usepackage{float}

\restylefloat{table}

\title{ECS 230 App Num Linear Algebra \\ Homework 2}
\author{Yuyang(Peter) Rong \\917781535 \\ PtrRong@ucdavis.edu}

\begin{document}

\maketitle

\begin{enumerate}
  \item \textbf{Simple Matrix Multiplication}
        Write a program that implements matrix multiplication $C=AB$ where $A$, $B$ and $C$ are $\mathbb{R}^{(n \times n)}$ real matrices represented as double-precision numbers.

        \textbf{Input Requirements}
        \begin{itemize}
          \item The value of $n$ must be read as a command line argument by the program.
          \item The program must use dynamic memory allocation for all matrices.
          \item The program must initialize the matrices $A$ and $B$ with non-zero arbitrary values of your choice before starting the calculation of the product.
        \end{itemize}

        \textbf{Return Requirements}
        \begin{itemize}
          \item Determine the total number of floating-point operations performed in the product.
          \item Use the  \lstinline$readTSC$ function(provided in Homework2) to measure the time taken to compute the product.
          \item Use repeated runs to ensure that timings are accurate.
        \end{itemize}

        \textbf{Print Requirements}
        The program must print the following information on output on two separate lines:
        \begin{itemize}
          \item the number of CPU clock cycles used (printed as an integer)
          \item the floating-point performance in GFlop/s, printed as a floating-point number.
        \end{itemize}

        \textbf{Solutions}

        a) Provide results for the following values of n: \SI{100}, \SI{200}, \SI{500}, \SI{1000}, \SI{2000}.

        \textbf{Running the Program}

        \begin{mdframed}
          % TODO
          Compilation: \lstinline$gcc -o output filename.c -flags$

          Running: \lstinline$./output n$

        \end{mdframed}


        \textbf{Formulas}

        \begin{mdframed}
          % Total number of floating point operation
          $$\text{num\_flops} = $$

          % Time taken to compute the product
          $$\text{total\_time} = $$

          % CPU Cycles Used
          $$\text{total\_cycles} = $$

          % Floating point performance GFlops/s
          $$\text{performance} = $$
        \end{mdframed}

        \begin{table}
          \centering
          \begin{tabular}{SSS}
            \toprule
            \multicolumn{1}{l}{Matrix Dimension (n)} &
            \multicolumn{1}{l}{Clock Cycles}         &
            \multicolumn{1}{l}{Gflops/s}                   \\
            \midrule
            100                                      &   & \\
            200                                      &   & \\
            500                                      &   & \\
            1000                                     &   & \\
            2000                                     &   & \\
            \bottomrule
          \end{tabular}
          \caption{Results of running program in subsection a}
        \end{table}


        \begin{figure}
          \centering
          \fbox{\rule[-.5cm]{0cm}{4cm} \rule[-.5cm]{4cm}{0cm}}
          \caption{Replace fbox with your graph. x axis = n = [100, 200, 500, 1000, 2000] yaxis = performance metrics}
        \end{figure}


        b) Comment on your results and compare with the expected floating-point performance of the processor (Intel Core i7-4790K, 4.0GHz or 3.6 GHz depending on the CSIF PC you are using. Note: determine the peak performance of one core of the processor considering the width of SIMD registers, clock frequency, hyper-threading, and the fused multiply-add FMA instruction).\textbf{ Compute the fraction of the nominal peak performance of a core achieved in your runs.}  Consider the fact that the “turbo boost” feature of Intel processors may change the frequency.

        \textbf{Formula}

        \begin{mdframed}
          $$\text{nominal\_peak\_performance} = $$
        \end{mdframed}

        \textbf{Analysis}

        \begin{mdframed}
          \vspace{1in}
        \end{mdframed}


  \item \textbf{Matrix Multiplication Loops}
        Write a modified version of the above program implementing all 6 possible loop orderings in a matrix multiplication.
        Run the program and compare the performance for each loop ordering. Discuss and explain your results.  Compare results obtained when compiling with \lstinline$gcc$ without options, and with the \lstinline$-O0$,\lstinline$-O2$ and \lstinline$-O3$ optimization options.

        \textbf{Solutions}

        \textbf{Running the Program}

        \begin{mdframed}
          % TODO
          Compilation: \lstinline{gcc -o output filename.c -flags}

          Running: \lstinline{./output n}

        \end{mdframed}

        \textbf{Algorithms}

        \begin{lstlisting}[language=C]
#include <stdio.h>
// Sample Loop
int main()
{
    printf("Hello world!");
    return 0;
}
\end{lstlisting}


        \begin{table}
          \centering
          \begin{tabular}{lSSS}
            \toprule
            \multicolumn{1}{l}{Loop Order} &
            \multicolumn{1}{l}{-O0}        &
            \multicolumn{1}{l}{-O2}        &
            \multicolumn{1}{l}{-O3}                 \\
            \midrule
            ijk                            &   &  & \\
            ikj                            &   &  & \\
            jik                            &   &  & \\
            jki                            &   &  & \\
            kij                            &   &  & \\
            kji                            &   &  & \\
            \bottomrule
          \end{tabular}
          \caption{Results of running different loops}
        \end{table}


        \begin{figure}
          \centering
          \fbox{\rule[-.5cm]{0cm}{4cm} \rule[-.5cm]{4cm}{0cm}}
          \caption{Replace fbox with your graph. x axis = n = [100, 200, 500, 1000, 2000] yaxis = performance of different loops}
        \end{figure}

  \item \textbf{BLAS Matrix Multiplication}
        Write a program implementing matrix multiplication using the  \lstinline$dgemm$ function of the BLAS library(http://www.netlib.org/blas). The library is installed on CSIF computers.  Use the command  \lstinline$man dgemm$ on CSIF to get details about the arguments of the function. Note that  \lstinline$dgemm$ is a FORTRAN function and all parameters from a C program should be passed through pointers. Use the \textit{dotblas.c} example program (provided on Canvas) as an example of how to call BLAS functions. Use the additional \lstinline$-lblas$ argument during compilation.

        \textbf{Measure timings for the above problem sizes and discuss your results.}

        Note: all programs in this homework assignment use only one core of the processor. However, the BLAS library is multithreaded.
        Use the command \lstinline$export OMP_NUM_THREADS=1$ before running your program to make sure that you measure the performance of one core only

        \textbf{Solutions}


        \textbf{Results of running different loops}

        \begin{table}
          \centering
          \begin{tabular}{SSS}
            \toprule
            \multicolumn{1}{l}{\lstinline{dgemm} dimension} &
            \multicolumn{1}{l}{Clock cycle}                 &
            \multicolumn{1}{l}{Total seconds}                     \\
            \midrule
            100                                             &   & \\
            200                                             &   & \\
            500                                             &   & \\
            1000                                            &   & \\
            2000                                            &   & \\
            \bottomrule
          \end{tabular}
          \caption{Some caption}

        \end{table}



        \begin{figure}
          \centering
          \fbox{\rule[-.5cm]{0cm}{4cm} \rule[-.5cm]{4cm}{0cm}}
          \caption{Replace fbox with your graph.  X axis is for different dimension and Y for clock cycle}
        \end{figure}


        \textbf{Running the Program}

        \begin{mdframed}
          % TODO
          Compilation: \lstinline$gcc -o output filename.c -flags$

          Running: \lstinline$./output n$

        \end{mdframed}

        \textbf{Formula}

        \begin{mdframed}
          $$\text{total\_seconds} = $$
        \end{mdframed}

        \textbf{Discussing the Results}

        \begin{mdframed}
          \vspace{3em}

        \end{mdframed}
\end{enumerate}

\end{document}